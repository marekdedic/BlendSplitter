A blender-\/like Qt Widget management library, version 3.

You can download the whole documentation in pdf on (\href{https://genabitu.github.io/BlendSplitter/latex/BlendSplitter.pdf}{\tt https\+://genabitu.\+github.\+io/\+Blend\+Splitter/latex/\+Blend\+Splitter.\+pdf}).

This library offers 2 kinds of functionality -\/ one implemented by the \hyperlink{class_blend_splitter}{Blend\+Splitter} class, the other by the \hyperlink{class_switching_widget}{Switching\+Widget} class. Although these are intended to be used together, each one of them can be used separately.

\section*{\hyperlink{class_blend_splitter}{Blend\+Splitter} }

This widget implements the functionality of Blender (Open-\/source 3D modelling software) widget management. This widget displays a splitter similar to Q\+Splitter. However, each widget in \hyperlink{class_blend_splitter}{Blend\+Splitter} has a pair of Expanders (one in top right and one in bottom left corner). By dragging from these Expanders inwards a new widget is created in the direction of the drag. If the direction is different to that of the \hyperlink{class_blend_splitter}{Blend\+Splitter}, a new \hyperlink{class_blend_splitter}{Blend\+Splitter} with parallel direction is created in place of the widget with the widget and the new widget in it. By dragging from these expanders outwards, a neighbouring widget (or a collection of widgets) can be closed. While the mouse is held, the widgets to be closed are marked with black overlay. When the mouse is released, they are closed. \hyperlink{class_blend_splitter}{Blend\+Splitter} can be used like any other Q\+Widget, although setting one as the central widget is recommended. A \hyperlink{class_blend_splitter}{Blend\+Splitter} can contain objects of any class inheriting from Q\+Widget. Note that you have to manually set the initial state of the \hyperlink{class_blend_splitter}{Blend\+Splitter}. You need to add at least 1 widget, otherwise nothing will be displayed.

\hyperlink{class_blend_splitter}{Blend\+Splitter} provides 3 static variables that allow some customization of the library design. These are expander\+Size, switching\+Bar\+Height and expander\+Image. These are all initialized with default values. The default Expander image is provided by the library.

The default Expander image\+:

 \subsection*{Example }


\begin{DoxyCode}
\{C++\}
#include <QApplication>
#include <QMainWindow>

#include <BlendSplitter>

int main(int argc, char** argv)
\{
    new QApplication\{argc, argv\};
    QMainWindow* window\{new QMainWindow\{\}\};
    BlendSplitter* splitter\{new BlendSplitter\{[]()->QWidget* \{return new QLabel\{"My Widget"\};\}\}\};

    window->setCentralWidget(splitter);
    window->resize(400, 200);
    window->setWindowTitle("BlendSplitter example");

    splitter->addWidget();

    window->show();
    return qApp->exec();
\}
\end{DoxyCode}
 On Gnome 3.\+22, this example looks like\+:

 \subsection*{Example with composite splitters }


\begin{DoxyCode}
\{C++\}
#include <QApplication>
#include <QMainWindow>

#include <BlendSplitter>

int main(int argc, char** argv)
\{
    new QApplication\{argc, argv\};
    QMainWindow* window\{new QMainWindow\{\}\};
    BlendSplitter* splitter\{new BlendSplitter\{[]()->QWidget* \{return new QLabel\{"My Widget"\};\}\}\};

    window->setCentralWidget(splitter);
    window->resize(400, 200);
    window->setWindowTitle("BlendSplitter example 2");

    splitter->addWidget();
    BlendSplitter* splitter2\{new BlendSplitter\{[]()->QWidget* \{return new QLabel\{"My Widget"\};\},
       Qt::Vertical\}\};
    splitter->addSplitter(splitter2);
    splitter2->addWidget();
    splitter2->addWidget();

    window->show();
    return qApp->exec();
\}
\end{DoxyCode}
 On Gnome 3.\+22, this example looks like\+:

 \section*{\hyperlink{class_switching_widget}{Switching\+Widget} }

This class displays a Widget with a \hyperlink{class_switching_bar}{Switching\+Bar} on the bottom. The widget displayed is one from \hyperlink{class_widget_registry}{Widget\+Registry} and it can be selected using a combo box in the \hyperlink{class_switching_bar}{Switching\+Bar}. The \hyperlink{class_switching_bar}{Switching\+Bar} is like a Q\+Menu\+Bar, but can also contain plain widgets. A \hyperlink{class_switching_widget}{Switching\+Widget} can contain objects of any class inheriting from Q\+Widget.

Note that constructing an object of this class when \hyperlink{class_widget_registry}{Widget\+Registry} is empty will cause a default \hyperlink{class_registry_item}{Registry\+Item} to be added to it. The height of the \hyperlink{class_switching_bar}{Switching\+Bar} can be modified by changing \hyperlink{class_blend_splitter_a478fa3cfcf59f76edf8f021bee297e0d}{Blend\+Splitter\+::switching\+Bar\+Height}.

\subsection*{Example }


\begin{DoxyCode}
\{C++\}
#include <QApplication>
#include <QMainWindow>

#include <BlendSplitter>

int main(int argc, char** argv)
\{
    new QApplication\{argc, argv\};
    QMainWindow* window\{new QMainWindow\{\}\};

    WidgetRegistry::getRegistry()->addItem();
    WidgetRegistry::getRegistry()->addItem("Type1", []()->QWidget* \{return new QLabel\{"Type 1 Label"\};\},
       [](SwitchingBar* bar, QWidget*)->void \{
        QMenu* menu\{new QMenu\{"My first menu"\}\};
        bar->addMenu(menu);
        QMenu* menu2\{new QMenu\{"My second menu"\}\};
        menu2->addAction(new QAction\{"New", 0\});
        menu2->addAction(new QAction\{"Close", 0\});
        bar->addMenu(menu2);
        QLabel* lab\{new QLabel\{"My third not-so-menu"\}\};
        bar->addWidget(lab);
    \});
    WidgetRegistry::getRegistry()->addItem(new RegistryItem\{"Type2", []()->QWidget* \{return new
       QLabel\{"Type 2 Label"\};\}\});
    WidgetRegistry::getRegistry()->setDefault(1);

    SwitchingWidget* widget\{new SwitchingWidget\{\}\};

    window->setCentralWidget(widget);
    window->resize(600, 400);
    window->setWindowTitle("SwitchingWidget example");

    window->show();
    return qApp->exec();
\}
\end{DoxyCode}
 On Gnome 3.\+22, this example looks like\+:

